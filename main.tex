\documentclass[fleqn]{jsarticle}

\title{サイバーセキュリティ演習 I}
\author{高野 祐輝}

\begin{document}

\maketitle

\section{目的と評価方法}

\begin{quote}
\begin{description}
    \item[最優] ファイアウォール技術とペネトレーションテストを組み合わせて,検疫ネットワークの設計と構築を行うことができる
    \item[優] ファイアウォール技術でDeMilitarized Zoneのあるネットワーク設計と構築を行うことができる
    \item[良] ファイアウォール技術で適切なネットワークアクセスコントロールができる
    \item[可] 各種サイバー攻撃手法と防御手法について論じることができる
\end{description}
\end{quote}


\section{セキュリティ哲学}

\subsection{サイバーセキュリティとは何か}

\subsection{サイバーキルチェーン}

\cite{hutchins2011intelligence}

\subsection{セキュリティポリシとユーザビリティ}

\cite{RFC2196}

\section{TCP/IPの基礎}

\begin{itemize}
    \item OSI参照モデル
    \item L2ヘッダ
    \item L3ヘッダ
    \item L4ヘッダ
\end{itemize}

\section{PF (Packet Filter)の基礎}

\section{パケットフィルタリング}

\section{攻撃手法と対策}

\section{DMZ(DeMilitarized Zone)の構築}

\section{検疫ネットワークの構築}

\section{ロードバランス}

\section{ロギング}

\appendix

\section{Vagrantによる実験環境の構築}

\section{PFの構文}


\bibliography{ref,rfc} %hoge.bibから拡張子を外した名前
\bibliographystyle{junsrt} %参考文献出力スタイル

\end{document}